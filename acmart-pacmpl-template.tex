%% For double-blind review submission
\documentclass[acmsmall,10pt,review,anonymous]{acmart}\settopmatter{printfolios=true}
%% For single-blind review submission
%\documentclass[acmsmall,10pt,review]{acmart}\settopmatter{printfolios=true}
%% For final camera-ready submission
%\documentclass[acmsmall,10pt]{acmart}\settopmatter{}

%% Note: Authors migrating a paper from PACMPL format to traditional
%% SIGPLAN proceedings format should change 'acmsmall' to
%% 'sigplan'.


%% Some recommended packages.
\usepackage{booktabs}   %% For formal tables:
                        %% http://ctan.org/pkg/booktabs
\usepackage{subcaption} %% For complex figures with subfigures/subcaptions
                        %% http://ctan.org/pkg/subcaption
\usepackage{mathtools}

\newcommand{\semrule}[4]{ #1 & \xrightarrow[m]{#2} & #3 & #4 }
\newcommand{\tuple}[1]{\langle #1 \rangle}
\newcommand{\case}[4]{\left \{ \begin{array}{lr} #1 & #2 \\ #3 & #4 \end{array} \right.}
\newcommand{\casethree}[6]{\left \{ \begin{array}{lr} #1 & #2 \\ #3 & #4 \\ #5 & #6 \end{array} \right.}


\makeatletter\if@ACM@journal\makeatother
%% Journal information (used by PACMPL format)
%% Supplied to authors by publisher for camera-ready submission
\acmJournal{PACMPL}
\acmVolume{1}
\acmNumber{1}
\acmArticle{1}
\acmYear{2017}
\acmMonth{1}
\acmDOI{10.1145/nnnnnnn.nnnnnnn}
\startPage{1}
\else\makeatother
%% Conference information (used by SIGPLAN proceedings format)
%% Supplied to authors by publisher for camera-ready submission
\acmConference[PL'17]{ACM SIGPLAN Conference on Programming Languages}{January 01--03, 2017}{New York, NY, USA}
\acmYear{2017}
\acmISBN{978-x-xxxx-xxxx-x/YY/MM}
\acmDOI{10.1145/nnnnnnn.nnnnnnn}
\startPage{1}
\fi


%% Copyright information
%% Supplied to authors (based on authors' rights management selection;
%% see authors.acm.org) by publisher for camera-ready submission
\setcopyright{none}             %% For review submission
%\setcopyright{acmcopyright}
%\setcopyright{acmlicensed}
%\setcopyright{rightsretained}
%\copyrightyear{2017}           %% If different from \acmYear


%% Bibliography style
\bibliographystyle{ACM-Reference-Format}
%% Citation style
%% Note: author/year citations are required for papers published as an
%% issue of PACMPL.
\citestyle{acmauthoryear}   %% For author/year citations



\begin{document}

%% Title information
\title[Serverless IFC]{Enforcing Dynamic Information Flow Control in Serverless Applications}         %% [Short Title] is optional;
                                        %% when present, will be used in
                                        %% header instead of Full Title.
% \titlenote{with title note}             %% \titlenote is optional;
                                        %% can be repeated if necessary;
                                        %% contents suppressed with 'anonymous'
% \subtitle{Subtitle}                     %% \subtitle is optional
% \subtitlenote{with subtitle note}       %% \subtitlenote is optional;
                                        %% can be repeated if necessary;
                                        %% contents suppressed with 'anonymous'


%% Author information
%% Contents and number of authors suppressed with 'anonymous'.
%% Each author should be introduced by \author, followed by
%% \authornote (optional), \orcid (optional), \affiliation, and
%% \email.
%% An author may have multiple affiliations and/or emails; repeat the
%% appropriate command.
%% Many elements are not rendered, but should be provided for metadata
%% extraction tools.

%% Author with single affiliation.
\author{First1 Last1}
\authornote{with author1 note}          %% \authornote is optional;
                                        %% can be repeated if necessary
\orcid{nnnn-nnnn-nnnn-nnnn}             %% \orcid is optional
\affiliation{
  \position{Position1}
  \department{Department1}              %% \department is recommended
  \institution{Institution1}            %% \institution is required
  \streetaddress{Street1 Address1}
  \city{City1}
  \state{State1}
  \postcode{Post-Code1}
  \country{Country1}
}
\email{first1.last1@inst1.edu}          %% \email is recommended

%% Author with two affiliations and emails.
\author{First2 Last2}
\authornote{with author2 note}          %% \authornote is optional;
                                        %% can be repeated if necessary
\orcid{nnnn-nnnn-nnnn-nnnn}             %% \orcid is optional
\affiliation{
  \position{Position2a}
  \department{Department2a}             %% \department is recommended
  \institution{Institution2a}           %% \institution is required
  \streetaddress{Street2a Address2a}
  \city{City2a}
  \state{State2a}
  \postcode{Post-Code2a}
  \country{Country2a}
}
\email{first2.last2@inst2a.com}         %% \email is recommended
\affiliation{
  \position{Position2b}
  \department{Department2b}             %% \department is recommended
  \institution{Institution2b}           %% \institution is required
  \streetaddress{Street3b Address2b}
  \city{City2b}
  \state{State2b}
  \postcode{Post-Code2b}
  \country{Country2b}
}
\email{first2.last2@inst2b.org}         %% \email is recommended


%% Paper note
%% The \thanks command may be used to create a "paper note" ---
%% similar to a title note or an author note, but not explicitly
%% associated with a particular element.  It will appear immediately
%% above the permission/copyright statement.
\thanks{with paper note}                %% \thanks is optional
                                        %% can be repeated if necesary
                                        %% contents suppressed with 'anonymous'


%% Abstract
%% Note: \begin{abstract}...\end{abstract} environment must come
%% before \maketitle command
\begin{abstract}
Text of abstract \ldots.
\end{abstract}


%% 2012 ACM Computing Classification System (CSS) concepts
%% Generate at 'http://dl.acm.org/ccs/ccs.cfm'.
\begin{CCSXML}
<ccs2012>
<concept>
<concept_id>10011007.10011006.10011008</concept_id>
<concept_desc>Software and its engineering~General programming languages</concept_desc>
<concept_significance>500</concept_significance>
</concept>
<concept>
<concept_id>10003456.10003457.10003521.10003525</concept_id>
<concept_desc>Social and professional topics~History of programming languages</concept_desc>
<concept_significance>300</concept_significance>
</concept>
</ccs2012>
\end{CCSXML}

\ccsdesc[500]{Software and its engineering~General programming languages}
\ccsdesc[300]{Social and professional topics~History of programming languages}
%% End of generated code


%% Keywords
%% comma separated list
\keywords{keyword1, keyword2, keyword3}  %% \keywords is optional


%% \maketitle
%% Note: \maketitle command must come after title commands, author
%% commands, abstract environment, Computing Classification System
%% environment and commands, and keywords command.
\maketitle


\section{Introduction}

Text of paper \ldots

\section{Semantics}

Tables~\ref{ta:ifc-total} and~\ref{ta:ifc-partial} summarize the 
proposed operational semantics for totally and parially ordered
security labels respectively.

\begin{table}
\[
\begin{array}{lclr}
\hline\hline
\semrule{\tuple{e,empty}}{w}{\tuple{e,e}}{\mbox{write to an empty cell in the data store}}\\
\hline
\semrule{\tuple{e,l}}{w}{\case{\mbox{error}}{\mbox{if} ~ e > l}{\tuple{e,e}}{\mbox{otherwise}}} {\mbox{write to a cell in the data store}}\\
\hline
\semrule{\tuple{e,empty}}{r}{\tuple{e,empty}, empty}{\mbox{read an empty cell in the data store}}\\
\hline
\semrule{\tuple{e,l}}{r}{\case{\tuple{max(e,l),l}}{\mbox{if} ~ m \geq l}{\tuple{e,l}, empty}{\mbox{otherwise}}} {\mbox{read to a cell in the data store}}\\
\hline\hline
\end{array}
\]
\caption{\label{ta:ifc-total}%
Small-step operational semantics for enforcing IFC with a total order over the security labels. $m$ is a paramater of the program execution s.t. m is the maximal security label allowed for the execution.}
\end{table}

\begin{table}
\[
\begin{array}{lclr}
\hline\hline
\semrule{\tuple{e,\emptyset}}{w(v)}{\tuple{e,\{\tuple{v,e}\}}}{\mbox{write to an empty cell in the data store}}\\
\hline
\semrule{\tuple{e,S}}
        {w(v)}
        {\case{\mbox{error}}
              {\exists \tuple{v',l'} \in S. ~ l' < e}
              {\tuple{e,(S \setminus \{\tuple{v',l'} | l' \geq e \}) \cup \{\tuple{v,e}\}}}
              {\mbox{otherwise}}}
        {\mbox{write to a cell in the data store}}\\
\hline
\semrule{\tuple{e,\emptyset}} {r} {\tuple{e,\emptyset}, empty} {\mbox{read an empty cell in the data store}}\\
\hline
\multicolumn{4} {l} {\mbox{let}~ S\!\downarrow_m = \{\tuple{v,l}\in S | l \leq m\} }\\
\semrule{\tuple{e,S}}
        {r}
        {\casethree{\tuple{e,S}, empty} {|S\!\downarrow_m| = 0}
                   {\tuple{max(e,l), S}, v} {|S\!\downarrow_m| = 1}
                   {\mbox{error}}{|S\!\downarrow_m| > 1}}
        {\mbox{read to a cell in the data store}}\\
\hline\hline
\end{array}
\]
\caption{\label{ta:ifc-partial}%
Small-step operational semantics for enforcing IFC with a partial order over the security labels. $m$ is a paramater of the program execution s.t. m is the maximal security label allowed for the execution.}
\end{table}

\section{Discussion}

We employ two techniques to enforce confidentiality at the data-store
level: (1) faceting and (2) the no-sensitive-upgrade (NSU) rule.  The former
restricts information flow by presenting different views of the store
to different observers.  We use it to isolate writes by incomparable 
users.  The latter prevents the flow of information down the lattice by
blocking high writes to low locations.  Both mechanisms preserve
termination sensitivity.  NSU reduces faceting (which can be seen as a
form of store corruption that requires manual intervention) by
sacrificing transparency.  Without NSU, we would need to introduce
facets for each label, not just incomparable ones.

\section{Properties of the Semantics}

Below we state some properties of our proposed scheme which need to be verified w.r.t.
a formal semantics of executions.

\begin{conj}
Our semantics enforces termination-sensitive noninterference, i.e., whenever there exists
a way to violate the information flow requirement, it enforces noninterference by either yielding an error (the no-sensitive-upgrade case)
or faceting the data store (the incomparable writes case).
\end{conj}

\begin{conj}
Our semantics can be easily implemented for Javascript programs using
any key value DB with very low runtime overhead.
\end{conj}

\begin{conj}
Our semantics can identify interesting bugs in real applications.
\end{conj}

\begin{conj}
Our semantics is not transparent, i.e., there exist correct programs for which our semantics
will yield error.
\end{conj}

\begin{conj}
In practical serverless applications, the no-sensitive-upgrade semantics is preferable to transparency
as it avoids data store corruption.
\end{conj}

\begin{conj}
Correct programs which are rejected by our semantics can be modified to work with our IFC model with modest implementation effort.
\end{conj}

%\section{Partial Order}


%% Acknowledgments
\begin{acks}                            %% acks environment is optional
                                        %% contents suppressed with 'anonymous'
  %% Commands \grantsponsor{<sponsorID>}{<name>}{<url>} and
  %% \grantnum[<url>]{<sponsorID>}{<number>} should be used to
  %% acknowledge financial support and will be used by metadata
  %% extraction tools.
  This material is based upon work supported by the
  \grantsponsor{GS100000001}{National Science
    Foundation}{http://dx.doi.org/10.13039/100000001} under Grant
  No.~\grantnum{GS100000001}{nnnnnnn} and Grant
  No.~\grantnum{GS100000001}{mmmmmmm}.  Any opinions, findings, and
  conclusions or recommendations expressed in this material are those
  of the author and do not necessarily reflect the views of the
  National Science Foundation.
\end{acks}


%% Bibliography
%\bibliography{bibfile}


%% Appendix
\appendix
\section{Appendix}

Text of appendix \ldots

\end{document}
