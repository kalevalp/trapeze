\documentclass{article}


\usepackage{amsmath}
\usepackage{amssymb}
\usepackage{color}
\usepackage[margin=2cm]{geometry}

\newcommand{\store}{\Delta}
\newcommand{\storeupdate}[3]{#1[#2\,{:}{=}\,#3]}
\newcommand{\storeaccess}[2]{#1(#2)}
\newcommand{\indomain}[2]{#1\in\text{\textsf{dom}}(#2)}
\newcommand{\notindomain}[2]{#1\notin\text{\textsf{dom}}(#2)}

\newcommand{\facet}[3]{\langle#1\,{?}\,#2\,{:}\,#3\rangle}
\newcommand{\ffacet}[3]{\langle\!\facet{#1}{#2}{#3}\!\rangle}
\newcommand{\splits}[4]{(\!\!\facet{\genfrac{}{}{0pt}{1}{#1}{#2}}{#3}{#4}\!\!)}

\newlength{\longrightarrowlength}
\settowidth{\longrightarrowlength}{$\longrightarrow$}
\newlength{\notlength}
\settowidth{\notlength}{$\not$}

\newcommand{\startrules}{\begin{array}{l@{~{,}~}lcl@{~{,}~}l|ll}}
\newcommand{\finishrules}{\end{array}}
\newcommand{\rn}[1]{\text{\textsc{[#1]}}}
\newcommand{\mrule}[3]{%
      \multicolumn{6}{c}{
        \dfrac{
          \begin{array}{c}
            #2
          \end{array}
        }{#3}
      }
    &
      \rn{#1}
  \\
      \multicolumn{7}{c}{}
  \\%
}
\newcommand{\steprule}[5]{%
      \multicolumn{2}{l}{ #3 }
    &
      {#4}
    &
      \multicolumn{2}{l|}{ #5 }
    &
      #2
    &
      \rn{#1}
  \\%
}
\newcommand{\tsteparrow}[1]{\overset{#1}{\longrightarrow}}
\newcommand{\tstep}[3]{#2\tsteparrow{#1}#3}
\newcommand{\tsteprule}[5]{\steprule{#1}{#2}{#4}{\tsteparrow{#3}}{#5}}
\newcommand{\ssteparrow}[1]{\overset{#1}{\longrightarrow}}
\newcommand{\ssteparrowstar}[1]{\overset{#1}{\longrightarrow^*}}
\newcommand{\sstep}[3]{#2\ssteparrow{#1}#3}
\newcommand{\sstepstar}[3]{#2\ssteparrowstar{#1}#3}
\newcommand{\ssteprule}[5]{\steprule{#1}{#2}{#4}{\ssteparrow{#3}}{#5}}
\newcommand{\ltsteparrow}[2]{\overset{#2}{\underset{#1}{\longrightarrow}}}
\newcommand{\ltstep}[4]{#3\ltsteparrow{#1}{#2}#4}
\newcommand{\ltsteprule}[6]{\steprule{#1}{#2}{#5}{\ltsteparrow{#3}{#4}}{#6}}
\newcommand{\lssteparrow}[2]{\overset{#2}{\underset{#1}{\longrightarrow}}}
\newcommand{\lsstep}[4]{#3\lssteparrow{#1}{#2}#4}
\newcommand{\lssteprule}[6]{\steprule{#1}{#2}{#5}{\lssteparrow{#3}{#4}}{#6}}

\newcommand{\hole}{{\bullet}}
\newcommand{\hash}[2]{\##1\,{:}{=}\,#2}
\newcommand{\replace}[3]{\storeupdate{#1}{#2}{#3}}
\newcommand{\upd}[3]{\replace{#1}{#2}{#3}}
\newcommand{\ret}[1]{\text{\textsf{return}}~#1}
\newcommand{\bindIO}[2]{%
  #1\,
  \overset{\text{\texttt{IO}}}
    {{>}\!\!{>}\!{=}}
  \,#2
  %
}
\newcommand{\bind}[2]{%
  #1\,
  \overset{\text{\texttt{FIO}}}
    {{>}\!\!{>}\!{=}}
  \,#2
  %
}
\newcommand{\newref}[1]{\text{\textsf{new}}~#1}
\newcommand{\readref}[1]{\text{\textsf{read}}~#1}
\newcommand{\writeref}[2]{\text{\textsf{write}}~#1~#2}
\newcommand{\Raw}[1]{\text{\textsf{Raw}}~#1}
\newcommand{\bindFac}[2]{\text{\textsf{BindFac}}~#1~#2}
%\newcommand{\bindFac}[2]{%
%  #1\,
%  \overset{\text{\texttt{Fac}}}
%    {{>}\!\!{>}\!{=}}
%  \,#2
%  %
%}
\newcommand{\prodd}[1]{\text{\textsf{runFacFIO}}~#1}
\newcommand{\threads}[3]{\,[\!\!\!\ffacet{#1}{#2}{#3}\!\!\!]\,}
\newcommand{\yield}[2]{\text{\textsf{yield}}~#1~#2}
\newcommand{\readfile}[1]{\text{\textsf{get}}~#1}
\newcommand{\readfilen}[2]{\text{\textsf{get}}~#1~#2}
\newcommand{\writefile}[2]{\text{\textsf{put}}~#1~#2}
\newcommand{\readevent}[2]{\text{\textsf{R}}(#1,#2)}
\newcommand{\writeevent}[2]{\text{\textsf{W}}(#1,#2)}
\newcommand{\notbr}[1]{\overline{#1}}
\newcommand{\concat}[2]{#1\,{{+}\!\!{+}}\,#2}

\newcommand{\letrec}[3]{\mu x{:}T.t}
\newcommand{\ifPositive}[3]{\text{\textsf{ifPositive $#1$ $#2$ $#3$}}}
\newcommand{\IntType}{\text{\textsf{Int}}}
\newcommand{\FacetedType}[1]{\text{\textsf{Faceted}}~#1}
\newcommand{\FIOType}[1]{\text{\textsf{FIO}}~#1}
\newcommand{\FIORefType}[1]{\text{\textsf{FIORef}}~#1}
\newcommand{\agree}[2]{#2\vdash#1}
\newcommand{\stdize}[1]{\s{standardize}{#1}}

\newcommand{\s}[1]{\text{\textit{#1}}}

\newcommand{\startsyntax}{\begin{array}{rclrll}}
\newcommand{\sepsyntax}{\\&&&|&}
\newcommand{\finishsyntax}{\end{array}}

\newcommand{\thread}[2]{(#1,#2)}
\newcommand{\opread}[2]{\textsf{read}~#1~#2}
\newcommand{\opwrite}[3]{\textsf{write}~#1~#2~#3}
\newcommand{\opsend}[3]{\textsf{send}~#1~#2~#3}
\newcommand{\opfork}[2]{\textsf{fork}~#1~#2}
\newcommand{\opstuck}{\textsf{stuck}}
\newcommand{\opraiselabel}[2]{\textsf{raiseLabel}~#1~#2}
\newcommand{\evstart}[1]{\textsf{input}~#1}
\newcommand{\evsend}[2]{\textsf{output}~#1~#2}
\newcommand{\proj}[2]{#1{\downarrow_{#2}}}
\newcommand{\lequiv}[3]{#2\approx_{#1}#3}


\begin{document}


$\startsyntax
  l     & \in & \s{Label}  \\
  ch    & \in & \s{Channel}  \\
  &           & \s{label} & : & \s{Channel}\to\s{Label}  \\
  v     & \in & \s{Value}  \\
  S     & \in & \s{LabeledValueSet} & = & \multicolumn{2}{l}{2^{\s{Value}\times\s{Label}}}  \\
  k     & \in & \s{Key}  \\
  \sigma& \in & \s{Store} & = & \s{Key}\to\s{LabeledValueSet}  \\
  c     & \in & \s{Code}  \\
  t     & \in & \s{Thread} & ::= & \thread{c}{l}  \\
  ts    & \in & \s{Threads} & = & \text{multisets of threads}  \\
  f     & \in & \s{ReadContinuation} & = & \s{LabeledValueSet}\to\s{Code}  \\
  op    & \in & \s{Operation} & ::= &
      \opread{k}{f}
    \sepsyntax
      \opwrite{k}{v}{c}
    \sepsyntax
      \opsend{ch}{v}{c}
    \sepsyntax
      \opfork{c}{c}
    \sepsyntax
      \opraiselabel{l}{c}
    \sepsyntax
      \opstuck
    \\
  e     & \in & \s{Event} & ::= &
      \evstart{t}
    \sepsyntax
      \evsend{ch}{v}
    \sepsyntax
      \epsilon
    \\
  es    & \in & \s{Events} & = & \s{Event}^*  \\
  \Sigma& \in & \s{State} & ::= & (\sigma,ts)  \\
  &           & \s{run} & : & \s{Code}\to\s{Operation}  \\
  &           & \s{write} & : & \s{LabeledValueSet}\times\s{Value}\times\s{Label}\to\s{LabeledValueSet}  \\
  &           & \s{write}(S,v,l) & = & (S\setminus\{(v',l')\in S~|~l\sqsubseteq l'\})\cup\{(v,l)\}  \\
  &           & \proj{\hole}{\hole}
                  & : & \text{``Various things''}\to\s{Label}\to\text{``Various things''}  \\
  &           & \proj{S}{l} & = & \{(v,l')\in S~|~l'\sqsubseteq l\}  \\
  &           & \proj{ts}{l} & = & \text{multiset}\{\thread{c}{l'}\in ts~|~l'\sqsubseteq l\}  \\
  &           & \proj{\sigma}{l} & = & \lambda k. \proj{\sigma(k)}{l}  \\
  &           & \proj{e}{l} & = &
    \begin{cases}
      e  & \exists c,l'.~e=\evstart{\thread{c}{l'}}\text{ and }l'\sqsubseteq l  \\
      e  & \exists ch,v.~e=\evsend{ch}{v}\text{ and }\s{label}(ch)\sqsubseteq l  \\
      \epsilon & \text{otherwise.}
    \end{cases}  \\
  &           & \proj{es}{l} & = & \text{\texttt{fmap}}~(\proj{}{l})~es  \\
  &           & \proj{(\sigma,ts)}{l} & = & (\proj{\sigma}{l},\proj{ts}{l})   \\
  &           & \lequiv{l}{\hole_1}{\hole_2} & \text{means} & \proj{\hole_1}{l}=\proj{\hole_2}{l}  \\
\finishsyntax$
\\
\fbox{$\tstep{e}{(\sigma,t)}{\Sigma}$} Thread transition relation
\\
$\startrules
  \tsteprule{send}
        {\s{run}(c)=\opsend{ch}{v}{c'}\qquad l\sqsubseteq\s{label}(ch)}
        {\evsend{ch}{v}}
        {(\sigma,\thread{c}{l})}
        {(\sigma,\{\thread{c'}{l}\})}
  \tsteprule{read}
        {\s{run}(c)=\opread{k}{f}}
        {\epsilon}
        {(\sigma,\thread{c}{l})}
        {(\sigma,\{\thread{f(\proj{\sigma(k)}{l})}{l}\})}
  \tsteprule{write}
        {\s{run}(c)=\opwrite{k}{v}{c'}}
        {\epsilon}
        {(\sigma,\thread{c}{l})}
        {(\upd{\sigma}{k}{\s{write}(\sigma(k),v,l)},\{\thread{c'}{l}\})}
  \tsteprule{fork}
        {\s{run}(c) = \opfork{c_1}{c_2}}
        {\epsilon}
        {(\sigma,\thread{c}{l})}
        {(\sigma,\{\thread{c_1}{l},\thread{c_2}{l}\})}
  \tsteprule{raise-label}
        {\s{run}(c) = \opraiselabel{l'}{c'}\qquad l\sqsubseteq l'}
        {\epsilon}
        {(\sigma,\thread{c}{l})}
        {(\sigma,\{\thread{c'}{l'}\})}
\finishrules$
\\
\fbox{$\sstep{e}{\Sigma}{\Sigma}$} State transition relation
\\
$\startrules
  \ssteprule{start}
        {}
        {\evstart{t}}
        {(\sigma,ts)}
        {(\sigma,ts+t)}
  \ssteprule{skip}
        {}
        {\epsilon}
        {(\sigma,ts)}
        {(\sigma,ts)}
  \ssteprule{thread}
        {\tstep{e}{(\sigma,\thread{c}{l})}{(\sigma',ts')}}
        {e}
        {(\sigma,ts\uplus\{\thread{c}{l}\})}
        {(\sigma',ts\uplus ts')}
\finishrules$
\\
\textbf{Theorem (TSNI).}
\\
If $\lequiv{l}{\Sigma_1}{\Sigma_2}$
and $\sstep{e_1}{\Sigma_1}{\Sigma_1'}$,
\\
then $\exists \Sigma_2'.\exists e_2.~\sstep{e_2}{\Sigma_2}{\Sigma_2'}$
and $\lequiv{l}{\Sigma_1'}{\Sigma_2'}$
and $\lequiv{l}{e_1}{e_2}$.
\\
\textit{Proof.}
Please continue reading below.
\\
\fbox{$\sstepstar{es}{\Sigma}{\Sigma}$} Multiple-step state transition relation
\\
$\startrules
  \mrule{}{
  }{
    \sstepstar{}{\Sigma}{\Sigma}
  }
  \mrule{}{
      \sstep{e}{\Sigma}{\Sigma'}
    \qquad
      \sstepstar{es}{\Sigma'}{\Sigma''}
  }{
    \sstepstar{e~es}{\Sigma}{\Sigma''}
  }
\finishrules$
\\
\textbf{Corollary (TSNI${}^*$).}
\\
If $\lequiv{l}{\Sigma_1}{\Sigma_2}$
and $\sstepstar{es_1}{\Sigma_1}{\Sigma_1'}$,
\\
then $\exists \Sigma_2'.\exists es_2.~\sstepstar{es_2}{\Sigma_2}{\Sigma_2'}$
and $\lequiv{l}{\Sigma_1'}{\Sigma_2'}$
and $\lequiv{l}{es_1}{es_2}$.
\\
\textit{Proof.}
By repeated application (i.e., use induction) of TSNI.


\newpage
\noindent
%
%
\fbox{$\ltstep{l}{e}{(\sigma,t)}{\Sigma}$} Projected thread transition relation
\\
$\startrules
  \ltsteprule{l-send}
        {\s{run}(c)=\opsend{ch}{v}{c'}\qquad l\sqsubseteq\s{label}(ch){\color{red}{}\sqsubseteq l'}}
        {{\color{red}l'}}
        {\evsend{ch}{v}}
        {(\sigma,\thread{c}{l})}
        {(\sigma,\{\thread{c'}{l}\})}
  \ltsteprule{l-censor}
        {\s{run}(c)=\opsend{ch}{v}{c'}\qquad l\sqsubseteq\s{label}(ch){\color{red}{}\not\sqsubseteq l'}}
        {{\color{red}l'}}
        {{\color{red}\epsilon}}
        {(\sigma,\thread{c}{l})}
        {(\sigma,\{\thread{c'}{l}\})}
  \ltsteprule{l-read}
        {\s{run}(c)=\opread{k}{f}}
        {{\color{red}l'}}
        {\epsilon}
        {(\sigma,\thread{c}{l})}
        {(\sigma,\{\thread{f(\proj{\sigma(k)}{l})}{l}\})}
  \ltsteprule{l-write}
        {\s{run}(c)=\opwrite{k}{v}{c'}}
        {{\color{red}l'}}
        {\epsilon}
        {(\sigma,\thread{c}{l})}
        {(\upd{\sigma}{k}{\s{write}(\sigma(k),v,l)},\{\thread{c'}{l}\})}
  \ltsteprule{l-fork}
        {\s{run}(c) = \opfork{c_1}{c_2}}
        {{\color{red}l'}}
        {\epsilon}
        {(\sigma,\thread{c}{l})}
        {(\sigma,\{\thread{c_1}{l},\thread{c_2}{l}\})}
  \ltsteprule{l-raise-label}
        {\s{run}(c) = \opraiselabel{l'}{c'}\qquad l\sqsubseteq l'{\color{red}{}\sqsubseteq l''}}
        {{\color{red}l''}}
        {\epsilon}
        {(\sigma,\thread{c}{l})}
        {(\sigma,\{\thread{c'}{l'}\})}
  \ltsteprule{l-censor-2}
        {\s{run}(c) = \opraiselabel{l'}{c'}\qquad l\sqsubseteq l'{\color{red}{}\not\sqsubseteq l''}}
        {{\color{red}l''}}
        {\epsilon}
        {(\sigma,\thread{c}{l})}
        {(\sigma,\{\})}
\finishrules$
\\
\fbox{$\lsstep{l}{e}{\Sigma}{\Sigma}$} Projected state transition relation
\\
$\startrules
  \lssteprule{l-start}
        {{\color{red}l\sqsubseteq l'}}
        {{\color{red}l'}}
        {\evstart{\thread{c}{l}}}
        {(\sigma, ts)}
        {(\sigma, ts+\thread{c}{l})}
  \lssteprule{l-skip}
        {}
        {{\color{red}l'}}
        {\epsilon}
        {(\sigma,ts)}
        {(\sigma,ts)}
  \lssteprule{l-thread}
        {\tstep{e}{(\sigma,\thread{c}{l})}{(\sigma',ts')}}
        {{\color{red}l'}}
        {e}
        {(\sigma,ts\uplus\{\thread{c}{l}\})}
        {(\sigma',ts\uplus ts')}
\finishrules$
\\
\textbf{Lemma (Invisibility).}
\\
If $l'\not\sqsubseteq l$
and $\tstep{e}{(\sigma,\thread{c}{l'})}{(\sigma',ts')}$,
\\
then $\proj{\sigma'}{l}=\proj{\sigma}{l}$
and $\proj{e}{l}=\epsilon$
and $\proj{ts'}{l}=\{\}$.
\\
\textit{Proof.}
Omitted.
\\
\textbf{Theorem (Projection 1).}
\\
\begin{tabular}{l@{$\qquad$}l}
  (1) & If $\sstep{e}{\Sigma}{\Sigma'}$,
\\
      & then $\lsstep{l}{\proj{e}{l}}{\proj{\Sigma}{l}}{\proj{\Sigma'}{l}}$.
\end{tabular}
\\
\textit{Proof.}
\\
\newcommand{\z}{$\quad$}
\begin{tabular}{l@{$\qquad$}l@{\qquad}l}
        & Proceed by cases (ie inversion) on (1).
\\
  (2)   & Case: $\rn{start}$.
\\
        & \z Let $\evstart{\thread{c}{l'}}=e$.
        & Justified by (2).
\\
        & \z Proceed by cases.
\\
        & \z Case where $l'\sqsubseteq l$.
\\
        & \z \z QED
        & by applying $\rn{l-start}$.
\\
        & \z Case where $l'\not\sqsubseteq l$.
\\
        & \z \z QED
        & by applying $\rn{l-skip}$.
\\
        & Case: $\rn{skip}$.
\\
        & \z QED
        & by applying $\rn{l-skip}$.
\\
  (2)   & Case: $\rn{thread}$.
\\
        & \z Let $c,l',\sigma',ts'$ be such that:
        & Justified by (2).
\\
  (3)   & \z \z $\tstep{e}{(\sigma,\thread{c}{l'})}{(\sigma',ts')}$
\\
        & \z Proceed by cases.
\\
  (4)   & \z Case where $l'\not\sqsubseteq l$.
\\
        & \z \z $\proj{\sigma'}{l}=\proj{\sigma}{l}$ and $\proj{e}{l}=\epsilon$ and $\proj{ts'}{l}=\{\}$
        & by Invisibility with (4) and (3).
\\
        & \z \z QED
        & by applying $\rn{l-skip}$.
\\
        & \z Case where $l'\sqsubseteq l$.
\\
        & \z \z Proceed by cases (ie inversion) on (3).
\\
  (4)   & \z \z Case: $\rn{send}$.
\\
        & \z \z \z Let $\evsend{ch}{v} = e$.
        & Justified by (4).
\\
        & \z \z \z Proceed by cases.
\\
        & \z \z \z Case where $label(ch)\sqsubseteq l$.
\\
        & \z \z \z \z QED
        & by applying $\rn{l-send}$.
\\
        & \z \z \z Case where $label(ch)\not\sqsubseteq l$.
\\
        & \z \z \z \z QED
        & by applying $\rn{l-censor}$.
\\
  (4)   & \z \z Case: $\rn{raise-label}$.
\\
        & \z \z \z Let $\opraiselabel{l''}{c'} = run(c)$.
        & Justified by (4).
\\
        & \z \z \z Proceed by cases.
\\
        & \z \z \z Case where $l''\sqsubseteq l$.
\\
        & \z \z \z \z QED
        & by applying $\rn{l-raise-label}$.
\\
        & \z \z \z Case where $l''\not\sqsubseteq l$.
\\
        & \z \z \z \z QED
        & by applying $\rn{l-censor-2}$.
\\
        & \z \z Cases: $\rn{read}$, $\rn{write}$, and $\rn{fork}$.
\\
        & \z \z \z Fairly straightforward.
\end{tabular}
\\
\textbf{Theorem (Projection 2).}
\\
\begin{tabular}{l@{$\qquad$}l}
  (1) & If $\lsstep{l}{e_1}{\proj{\Sigma}{l}}{\Sigma_1}$,
\\
      & then $\exists \Sigma_2.\exists e_2.~\sstep{e_2}{\Sigma}{\Sigma_2}$
        and $\Sigma_1 = \proj{\Sigma_2}{l}$
        and $e_1 = \proj{e_2}{l}$.
\end{tabular}
\\
\textit{Proof.}
\\
\begin{tabular}{l@{$\qquad$}l@{\qquad}l}
        & Let $(\sigma,ts)=\Sigma$.
\\
        & Proceed by cases (ie inversion) on (1).
\\
  (2)   & Case: $\rn{l-start}$.
\\
        & \z Let $\evstart{\thread{c}{l'}}=e_1$.
        & Justified by (2).
\\
        & \z Pick $\Sigma_2 = (\sigma,ts+\thread{c}{l'})$ and $e_2=e_1$.
\\
        & \z QED
        & by applying $\rn{start}$.
\\
        & \multicolumn{2}{l}{\z (Fairly straightforward because $\rn{start}$ is the same as $\rn{l-start}$, except without the side condition.)}
\\
        & Case: $\rn{l-skip}$.
\\
        & \multicolumn{2}{l}{\z (Fairly straightforward because $\rn{skip}$ is the same as $\rn{l-skip}$.)}
\\
  (2)   & Case: $\rn{l-thread}$.
\\
        & \z Let $c,l',\sigma_1,ts_1$ be such that:
        & Justified by (2).
\\
  (3)   & \z \z $\thread{c}{l'}\in\proj{ts}{l}$
\\
  (4)   & \z \z $\ltstep{l}{e_1}{(\sigma,\thread{c}{l'})}{(\sigma_1,ts_1)}$
\\
        & \z $l'\sqsubseteq l$
        & due to (3) and the definition of $\proj{\hole}{\hole}$.
\\
        & \z Proceed by cases (ie inversion) on (4).
\\
  (5)   & \z Cases: $\rn{l-send}$ and $\rn{l-censor}$.
\\
        & \z \z Let $\opsend{ch}{v}{c'}=run(c)$.
        & Justified by (5).
\\
        & \z \z Pick $\Sigma_2 = (\sigma,ts-\thread{c}{l'}+\thread{c'}{l'})$
\\
        & \z \z and pick $e_2 = \evsend{ch}{v}$.
\\
        & \z \z QED
        & by applying $\rn{send}$.
\\
  (5)   & \z Cases: $\rn{l-raise-label}$ and $\rn{l-censor-2}$.
\\
        & \z \z Let $\opraiselabel{l''}{c'}=run(c)$.
        & Justified by (5).
\\
        & \z \z Pick $\Sigma_2 = (\sigma,ts-\thread{c}{l'}+\thread{c'}{l''})$
\\
        & \z \z and pick $e_2 = \epsilon$.
\\
        & \z \z QED
        & by applying $\rn{raise-label}$.
\\
        & \z Cases: $\rn{l-read}$, $\rn{l-write}$, and $\rn{l-fork}$.
\\
        & \multicolumn{2}{l}{\z \z (Fairly straightforward because the corresponding rules are the same.)}
\end{tabular}
\\
\textbf{Theorem (TSNI).}
\\
\begin{tabular}{l@{$\qquad$}l}
  (1) & If $\lequiv{l}{\Sigma_1}{\Sigma_2}$
\\
  (2) & and $\sstep{e_1}{\Sigma_1}{\Sigma_1'}$,
\\
      & then $\exists \Sigma_2'.\exists e_2.~\sstep{e_2}{\Sigma_2}{\Sigma_2'}$
        and $\lequiv{l}{\Sigma_1'}{\Sigma_2'}$
        and $\lequiv{l}{e_1}{e_2}$.
\end{tabular}
\\
\textit{Proof.}
\\
\begin{tabular}{l@{$\qquad$}l@{\qquad}l}
  (3) & $\lsstep{l}{\proj{e_1}{l}}{\proj{\Sigma_1}{l}}{\proj{\Sigma_1'}{l}}$
      & by applying Projection 1 to (2).
\\
  (4) & $\lsstep{l}{\proj{e_1}{l}}{\proj{\Sigma_2}{l}}{\proj{\Sigma_1'}{l}}$
      & by substituting (1) in (3).
\\
      & QED
      & by applying Projection 2 to (4).
\end{tabular}


\end{document}
